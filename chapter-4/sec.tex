\section{Making Decisions with Conditions}

\begin{itemize}
\item \textbf{nil} is equal to \textbf{`()} and both of them will be evaluating as a \textbf{false};
\item Any value not equivalent to an empty list will be considered as a true value;
\item The function "if":
  \subitem (if \textit{condition} \textit{reduce-case-condition-true} \textit{reduce-case-condition-false})
\item With \textbf{when} command, all the enclosed expressions are evaluated when the condition is true;
\item With \textbf{unless} command, all the enclosed expression are evaluated when the condition is false;
\item \textbf{cond} form is a classic way to do branching in Lisp:
  \subitem use a implicit \textbf{progn}, evaluated each expression and choose the first true sentence one;
\item \textbf{case} form is the classic \textbf{switch-case} in C-like programming;
\item In Comparing, we should use \textbf{Conrad rules}:
  \subitem 1. Use \textbf{eq} to compare symbols;
  \subitem 2. Use \textbf{equal} for evrything else;
\end{itemize}
